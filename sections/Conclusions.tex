\section{Conclusions}
\label{sec:conclusions}
BTSs account for most of a cellular network's energy consumption. Motivated by the non load-proportionality of BTS energy consumption, prior work proposed shutting down some BTSs when traffic is low. However, network operators are reluctant to do so for a variety of reasons.

To reduce energy consumption, we propose using a commonly available and used feature called BTS power savings that deactivates some TRXs at BTSs that have low traffic. Furthermore, calls may be handed-off from BTSs with higher load to neighboring ones with lighter load to increase the benefits of BTS power-saving.

Using real network topology and traffic traces in a simulation study , we found that merely using BTS power saving in an urban setting can result in considerable energy savings. Moreover, our results also indicate that periodic call-shuffling between BTSs can further reduce energy consumption in existing large GSM networks.