\section{Related Work}
\label{sec:related} 
The problem that we investigated in this work falls into the broad category of resource scheduling problem under constraints. Resource scheduling problem occurs in many different domains whereby similar solution strategies are applicable. Some examples of prior work in other domains on similar problems include resource provisioning in data centers~\cite{Jeyarani:2012:DIA:2148243.2148374,serverEnergy,Mazzucco:Maximizing:2011:CoRR,Oh:2011:ECS:2170444.2170458,Chase:2001:MES:502059.502045}, scheduling in compute clusters~\cite{AlDaoud2012745}, System on Chip (SOC)~\cite{Fang:2011:COP:1995896.1995940}, electric power systems and smart grid~\cite{Javed:2008:ULP:1485753.1485792,Logenthiran2011138,Celli:2001:PICA,FahadJavedAdOpt.SASO.2009.26}, WiFi access points~\cite{Marsan:2010:SAM:1791314.1791340}, wide area networks~\cite{Cavdar:2011:ECOC}, high performance computing~\cite{Lee:ServerConsolidation:2011:Globecom,Pinheiro01loadbalancing,Yao:DCPowerReduction:2012:INFOCOM,Herodotou:Starfish:2011:CIDR,Herodotou:2011:NOS:2038916.2038934,Aikema:ElecCostHPC:2011:ISSST} as well as cellular networks~\cite{Peng:2011:TPS:2030613.2030628,Peng:2011:BTSSaving:Mobicom}.

Our focus in the present work is to develop techniques that are practically usable in operational networks instead of taking a clean-slate approach. In this sense, it differs from much prior work in energy efficiency in cellular networks. A key benefit of this approach is that it 
does not require any additional hardware
and works within the GSM specifications. Our
work is very similar in spirit to the concept of
\textit{frequency dimming}
in~\cite{Tipper:Dimming:Globecom:2010} albeit at a different
level of abstraction. A similar approach is also proposed
in~\cite{Blume:2010:BLTJ:CellularPower} with some rough
estimates of expected savings. We, on the other hand, use site
locations and traffic traces from a large cellular network with
more than 13 million subscribers to run a simulation study
assessing the benefits of dynamic equipment scaling coupled
with call hand-offs. 